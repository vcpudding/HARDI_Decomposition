\documentclass[10pt]{article}
\usepackage[margin=1in]{geometry}

\usepackage[shortlabels]{enumitem}
\setcounter{secnumdepth}{4}


\usepackage{mathptmx}
\usepackage{graphicx}
\usepackage{times}
\usepackage{comment}
\usepackage{amstext}
\usepackage{amsmath}
\usepackage{amssymb}
\usepackage{array}
\usepackage{multirow}
\usepackage{url}
\usepackage{subfigure}
\usepackage{xcolor}
\usepackage{float}
\usepackage{slashbox}
\usepackage{pict2e}
\usepackage{tabularx}


\usepackage{fancyhdr, lastpage}
\pagestyle{fancy}
\fancyhf{}
%
\lhead{}
\chead{Comparison of denoising methods}
\rhead{}
%
\cfoot{Page \thepage{} of \protect\pageref*{LastPage}}

\usepackage{varioref}
\labelformat{equation}{(#1)}

% \usepackage{hyperref} must almost always be LAST \usepackage in
% preamble. Otherwise, you may get strange compilation errors!
\usepackage[colorlinks,linkcolor=black]{hyperref}

\begin{document}

Steps:

\begin{enumerate}
\item Start with $\delta=10$, estimate fiber directions with $M=3$.
\item Perform 5 trials with random initialization. Keep the results with convergence energy $e$ below the predefined threshold $t$ and discard others. We call the results with $e<t$ ``valid''.
\item If the number of valid results is below 3, increase $\delta$ by 10 and perform Step 2 again.
\item The number of fibers $M*$ is determined by the minimum number of fibers among the valid results. 
\item If there're more than one result such that the estimated number of fibers is equal to $M*$, take the mean of their directions and fractions as the final output.
\end{enumerate}

\end{document}


